\documentclass[12pt]{article}
\usepackage[italian]{babel}

\title{Metodi Informatici per la Gestione Aziendale}
\author{Giuseppe Facchi}
\date{A.A. 2020-2021}

\begin{document}
\maketitle
\newpage
\tableofcontents
\newpage

\section{Introduzione}
\subsection{Contenuti del corso}
\begin{itemize}
    \item Elementi di Economia e Organizzazione Aziendale
    \item Contabilità e Bilancio
    \item Finanza Aziendale
    \item Marketing Analytics
    \item Recommender systems
          \begin{itemize}
              \item Prediction Version of Problem
              \item Ranking Version of Problem (TOP-K Ranking)
          \end{itemize}
\end{itemize}

\paragraph{Economia}
\begin{itemize}
    \item \textbf{Microeconomia}: Produttori, Consumatori, Mercati
    \item Domanda e Offerta
    \item Incertezza e Comportamento del Consumatore
    \item Platform Economy e Sharing Economy
\end{itemize}

\subsection{Tipi di Economia}
Esistono \textbf{business models} diversi

\subsection{ICT e Business Disruption}
\begin{itemize}
    \item Nelle aziende i processi più importanti diventano mediati digitalmente
    \item Si parla di \textbf{landscape frammentato} e ricomposto in modi imprevedibili
    \item Le piattaforme ICT di relazione/negoziazione iniziano a proporsi per servizi professionali e alla professionali
    \item Nel \textbf{retail} anche la \textbf{customer experience} diventa \textbf{difitally mediated}
\end{itemize}
\newpage
\section{Elementi di economia e org. aziendale}
\subsection{La natura e i fini economi dell'impresa}
\paragraph{Attività economica} L'insieme delle operazioni finalizzate alla produzione, distribuzione o al consumo di \textbf{beni economici}
\paragraph{Beni economici} Beni utili a soddisfare i bisogni delle persone e SCARSI (non illimitati)

\paragraph{}\textit{Non tutti i bisogni delle persone sono affidati a imprese (es. sicurezza e giustizia), esistono però compromessi come la sanità}

\paragraph{} Quando si realizzano queste condizioni:
\begin{itemize}
    \item Esistenza di una \textbf{domanda} e \textbf{offerta}
    \item Esistenza di \textbf{soggetti} in grado di \textbf{offrire} questo \textbf{prodotto} agli acquirenti
    \item \textbf{Acquisizione} e \textbf{organizzazione} dei \textbf{fattori di produzione}
\end{itemize}
Allora la \textbf{progettazione/produzione} e \textbf{vendita} di \textbf{beni e servizi} sono realizzate a una particolare forma di organizzazione umana chiamata \textbf{impresa}.

\paragraph{Prezzi di vendita} Ai prezzi di vendita (accettabili per gli acquirenti) devono corrispondere \textbf{ricavi adeguati} a remunerare i costi di produzione.

\subsection{Gli Istituti Economici}
Quando \textbf{l'attività economica} è svolta attraverso realtà:
\begin{itemize}
    \item \textbf{Organizzate}
    \item \textbf{Durature}
    \item \textbf{Autonome}
\end{itemize}
si è in presenza di \textbf{istituti economici}
\begin{itemize}
    \item Famiglie
    \item Enti no-profit
    \item Amministrazioni pubbliche
    \item \textbf{imprese}
\end{itemize}

\subsubsection{L'istituto "impresa"}
Si caratterizza per la \textbf{specializzazione} in quella prate di \textbf{attività economica} che si manifesta attraverso la produzione e vendita sul \textbf{mercato} di prodotti \textit{(Beni/Servizi)}
\begin{itemize}
    \item \textbf{Combina} e \textbf{Trasforma} le risorse per realizzare prodotti di \textbf{maggior utilità e valore}
    \item Il processo di produzione \textbf{genera ricchezza} poiché \textbf{accresce il valore} (in termini monetizzabili) \textbf{finale dell'output} rispetto all'input
\end{itemize}

\subsection{Oggetto di analisi dell'ec. aziendale}
L'impresa è analizzabile come \textbf{combinazione di fattori produttivi}
\begin{itemize}
    \item \textbf{Lavoro}
    \item \textbf{Capitale}
\end{itemize}

\paragraph{}\textit{La \textbf{proprietà intellettuale} riveste un ruolo decisivo nella generazione del CASH-FLOW di un'azienda $\rightarrow$ \textbf{Immobilizzazioni Immateriali}}
\newline
\newline
L'\textbf{economia aziendale}
\begin{itemize}
    \item \textbf{Rileva}
    \item \textbf{Analizza}
    \item \textbf{Descrive}
\end{itemize}
tutte le \textbf{operazioni} pertinenti alle condizioni di produzione e in generale all'attività economica
\begin{itemize}
    \item Trasformazione Fisico-Tecnica
    \item Negoziazione di beni e servizi
    \item Organizzazione
    \item Rilevazione e informazione
\end{itemize}

\subsection{L'impresa come sistema}
L'impresa è un \textbf{sistema} in quanto è costituita da \textbf{diversi elementi} legati da relazione di \textbf{interdipendenza} e \textbf{combinati} tra loro per il raggiungimento di obiettivi prefissati. L'impresa è un sistema di tipo
\begin{itemize}
    \item \textbf{APERTO}: scambia input e output con l'ambiente in cui opera
    \item \textbf{DINAMICO}: modifica continuamente la sua configurazione
    \item \textbf{COMPLESSO}: ogni attività genera osservazioni di un elevano numero di variabili
    \item \textbf{UNITARIO}: Un corpo organico unico e unico mosso da un fine comune $\rightarrow$ \textit{Sistema decisionale TOP-DOWN}
\end{itemize}

\subsubsection{Sistema Aperto}
L'impresa nel proprio ambiente di riferimento \textbf{interagisce, scambia} continuamente risorse ed energia con altri attori secondo un modello \textbf{input-output}
\begin{itemize}
    \item \textbf{INPUT}: Importazione dall'esterno di fattori di produzione
    \item \textbf{OUTPUT}: Trasformazione dell'input e restituzione all'ambiente in forma di prodotti/servizi
\end{itemize}

\paragraph{}\textit{Esistono anche modelli \textbf{chiusi} (es. closed innovation system) dove esiste un solo input ed hanno una sola via d'uscita verso il mercato (es. Big Pharma) a cui vengono contrapposti sistemi \textbf{aperti} (es. Open Innovation Paradigm)}

\subsubsection{Sistema Dinamico}
Durante l'\textbf{attività ciclica} diu scambio con l'ambiente l'impresa \textbf{modifica costantemente} il suo assetto. Tende ad uno \textbf{stato di equilibrio dinamico} con il suo \textbf{ambiente} e al mutare delle condizioni esterne tende a mutare al suo interno con \textbf{procedure di retroazione} fino ad un nuovo equilibrio.

\subsubsection{Sistema Unitario}
L'impresa deve essere considerata come un corpo \textbf{unico} e \textbf{unito}

\subsection{L'impresa come attore sociale responsabile}
L'impresa si relaziona continuamente con gli \textbf{stakeholders}
\begin{itemize}
    \item Clienti
    \item Organizzazioni civili, politiche, organi legislativi
    \item Organi governativi
    \item Gruppi tutelatori degli interessi dei cittadini/lavoratori
\end{itemize}
La \textbf{responsabilità sociale} dell'impresa intende sottolineare la necessità per l'impresa di tener conto degli \textbf{interessi degli stakeholders} nelle sue scelte/azioni. Le aziende che riescono nell'intento vengono chiamate \textbf{Benefit Companies}

\subsection{Spazio decisionale dell'impresa}
\begin{itemize}
    \item \textbf{Incertezza}: Le decisioni strategiche fanno riferimento ad \textbf{archi temporali lunghi} in cui dominano \textbf{incertezze sui costi} dei fattori di produzione, \textbf{incertezze sui riferimenti amministrativi/istituzionali} e \textbf{incertezze sulla domanda} dei prodotti attuali
    \item \textbf{Diversi interessi da bilanciare}
    \item \textbf{Diversi decisori}: Contrasti tra \textbf{management} e \textbf{azionisti}
\end{itemize}
La comunità locale dove un'impresa industriale ha un grande stabilimento può trovarsi a \textbf{sopportare notevoli esternalità} dovute alla sua presenza
\begin{itemize}
    \item \textbf{Esternalità negative} (es. inquinamento)
    \item \textbf{Esternalità positive} (es. promozione di attività economiche dell'indotto)
\end{itemize}
Esiste anche una consapevolezza nel dover \textbf{gestire la responsabilità sociale (benefit companies)}
\begin{itemize}
    \item Bilancio ambientale
    \item Carte dei valori/Codici di comportamento
    \item Certificazioni sulla qualità
    \item Correttezza nella gestione del personale
\end{itemize}

\subsection{Principi base per il governo delle imprese}

\begin{itemize}
    \item Le \textbf{responsabilità} e il \textbf{potere di governo} spettano a coloro che rappresentano il \textbf{soggetto economico} (coincide spesso con i detentori del capitale sociale)
    \item Il perseguimento esclusivo del proprio interesse non può costituire il principio assoluto che orienta le decisioni del soggetto economico, bisogna infatti \textbf{rispettare le condizioni di equilibrio di medio/lungo termine} tra diverse grandezze economiche e tra gli stakeholders
\end{itemize}
Affinché le imprese possano risultare \textbf{durature} e aspirare al raggiungimento del loro \textbf{fine istituzionale} il \textbf{processo decisionale} del soggetto economico deve svolgersi sul rispetto di \textbf{due principi di governo}
\begin{itemize}
    \item \textbf{Principio di Economicità} che attiene alla natura economica del suo agire
    \item \textbf{Principio del Contemperamento degli Interessi} che fa riferimento al fine per cui l'impresa è stata costituita e opera (non solo il profitto)
\end{itemize}

\paragraph{Rule of Law} Certezza del diritto
\paragraph{Sostenibilità Ambientale} L'impresa deve essere governata seguendo il principio di sostenibilità ambientale
\paragraph{Equal Opportunity Employer} Tutela delle minoranze

\subsection{Il principio di economicità}
Il \textbf{soggetto economico} deve decidere e agire facendo in modo di perseguire per l'impresa:
\begin{itemize}
    \item \textbf{Equilibrio reddituale}: Equilibiro tra i componenti positivi (RICAVI) e negativi (COSTI) $\rightarrow$ Ricavi $>$ Costi
    \item \textbf{Equilibrio monetario}: Le entrate monetarie devono essere sufficienti per far fronte alle loro uscite anche ricorrendo ad una gestione \textit{finanziaria} $\rightarrow$ Entrate $>$ Uscite
    \item \textbf{Livello acettabile} di
          \begin{itemize}
              \item Efficienza (minimizzare gli sprechi)
              \item Flessibilità dei processi produttivi
          \end{itemize}
    \item \textbf{Congruità dei prezzi dei fattori di produzione acquisiti e dei prodotti venduti}
\end{itemize}

L'\textbf{economicità} delle imprese viene valutata attraverso l'analisi dei valori aziendali riferiti a un \textbf{determinato e significativo intervallo temporale} detto \textbf{esercizio}. I valori vengono rilevati mediante \textbf{metodologia contabile} e valutati attraverso il \textbf{bilancio di esercizio}. La \textbf{gestione finanziaria} ha il compito di mantenere un corretto equilibrio tra le diverse fonti di finanziamento dell'impresa e le modalità di impiego dei relativi mezzi finanziari.
L'attività di \textbf{determinazione degli obiettivi economici e di monitoraggio} del loro raggiungimento viene svolta attraverso \textbf{sistemi di programmazione controllo di gestione}.
\newpage
\subsection{L'assetto organizzativo dell'impresa}
L'impresa deve essere
\begin{itemize}
    \item \textbf{ORDINATA} nelle sue attività economiche
    \item \textbf{COORDINATA} nelle sue componenti
    \item \textbf{REGOLATA} nel suo funzionamento
\end{itemize}
L'\textbf{assetto organizzativo} è il risultato della combinazione di varaibili organizzative
\begin{itemize}
    \item \textbf{Struttura organizzativa} che definisce \textit{compiti e responsabilità}
    \item \textbf{Sistemi operativi (o di management)} che regolano e guidano i \textit{comportamenti} delle persone nello \textit{svolgere attività}
\end{itemize}
e ha il compito di \textbf{coniugare} il \textbf{lavoro} delle persone con il \textbf{sistema delle operazioni aziendali elementari} a loro volta combinate in \textbf{attività parziali} (Taylorismo)
\newpage
\section{La struttura organizzativa}

\end{document}