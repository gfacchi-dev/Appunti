\documentclass[12pt]{article}

\title{Sommario}
\author{}
\date{}

\begin{document}
    \maketitle
    \section{Introduzione alle Basi di Dati}
    \paragraph{Base di dati}
    \begin{itemize}
        \item Insieme organizzato di dati utilizzati per lo svolgimento di determinate attività (di un ente, azienda, ufficio, persona)
        \item Una risorsa integrata, condivisa fra i vari settori di una organizzazione
    \end{itemize}
    \paragraph{Struttura di un sistema informatico}
    \begin{itemize}
        \item Interfaccia utente
        \item Applicazioni
        \item Software di ambiente e di sistema
        \item Basi di dati
        \item Software di base 
        \item Hardware centralizzato e distribuito
        \item Sistema di comunicazione di rete
    \end{itemize}
    \paragraph{Ridondanza}
    Informazioni ripetute
    \paragraph{Rischio di incoerenza}
    Le versioni possono non coincidere
    \newpage
    \paragraph{Data Base Management System, Sistema di Gestione di Basi di Dati (DBMS)}
    Sistema (prodotto software) in grado di gestire collezioni di dati che siano (anche): 
    \begin{itemize}
        \item \textbf{Grandi}, di dimensioni (molto) maggiori della memoria centrale dei sistemi di calcolo utilizzati
        \item \textbf{Persistenti}, con un periodo di vita indipendente dalle singole esecuzioni dei programmi che le utilizzano
        \item \textbf{Condivise}, utilizzate da applicazioni diverse
    \end{itemize}
    e garantendo:
    \begin{itemize}
        \item \textbf{Affidabilità}, resistenza a malfunzionamenti hardware e software 
        \item \textbf{Sicurezza}, con una disciplina e un controllo degli accessi
        \item \textbf{Efficienza}, utilizzando al meglio le risorse di spazio e tempo del sistema
    \end{itemize}

    \paragraph{Schema (di base di dati)}
    Uno schema descrive un insieme di concetti, cioe’ classi di dati
    \paragraph{Istanza (di base di dati)}
    Una istanza descrive un insieme di dati che fanno riferimento ai concetti dello schema
    \paragraph{Modello}
    Insieme di costrutti utilizzati per organizzare i dati di interesse in concetti e descriverne la dinamica, cioè come cambiano nel tempo 
    \paragraph{Struttura di rappresentazione}
    Componenti fondamentali di un modello \textit{(es. Struttura utilizzata: la relazione)}
    \paragraph{Modello logico}
    \begin{itemize}
        \item utilizzati nei DBMS esistenti in commercio per l’organizzazione dei dati
        \item utilizzati dai programmi
        \item \textbf{indipendenti dalle strutture fisiche}
    \end{itemize}
	esempi: Relazionale,  reticolare, gerarchico, a oggetti
    \newpage
    \paragraph{Modello concettuale}
    permettono di rappresentare i dati in modo indipendente da ogni sistema DBMS
    \begin{itemize}
        \item cercano di descrivere direttamente i concetti del mondo reale
        \item sono utilizzati nelle fasi preliminari di progettazione
    \end{itemize}
	Il più noto è il modello Entità-Relazione
    \paragraph{Schema logico}
    Descrizione della base di dati con strutture di alto livello (ad esempio, la struttura di relazione)
    \paragraph{Schema fisico}
    Rappresentazione dello schema logico per mezzo di strutture fisiche di memorizzazione (files), blocchi in memoria, ecc.
    \paragraph{Schema esterno}
    Descrizione di parte della base di dati in un modello logico (“viste” parziali, di interesse per particolari gruppi di utenti)
    \paragraph{Indipendenza fisica}
    \textit{Il livello logico e quello esterno sono indipendenti da quello fisico}
    \begin{itemize}
        \item una relazione è utilizzata nello stesso modo qualunque sia la sua realizzazione fisica
        \item la realizzazione fisica può cambiare senza che debbano essere modificati i programmi software
    \end{itemize}
    \paragraph{Indipendenza logica}
    \textit{Il livello esterno è indipendente da quello logico}: 
    \begin{itemize}
        \item aggiunte o modifiche agli schemi esterni non richiedono modifiche al livello logico
        \item modifiche allo schema logico che lascino inalterato lo schema esterno sono trasparenti 
    \end{itemize}

    \paragraph{Data Description Language}
    Linguaggio di descrizione dei dati, con cui si descrivono le strutture degli schemi
    \paragraph{Data Manipulation Language}
    Linguaggio di manipolazione dei dati, con cui si esprimono le interrogazioni che ritrovano i dati dalla base di dati e le transazioni che li aggiornano (es. SQL)

    \section{Il modello Entità-Relazione}
    \paragraph{Entità}
    Una entità è una classe di oggetti (fatti, persone, cose) che:
    \begin{itemize}
        \item sono di interesse per l’applicazione
        \item hanno esistenza autonoma
        \item hanno proprietà comuni
    \end{itemize}
    \paragraph{Attributo di Entità}
    Un attributo di entità è una proprietà locale di un’entità, di interesse ai fini dell’applicazione, cioè una proprietà il cui valore in ogni istanza della entità dipende solamente dall’istanza della entità, e non da altri elementi dello schema
    \paragraph{Dominio di attributo}
    Si possono utilizzare come domini quelli tipicamente definiti nei linguaggi programmativi
    \paragraph{Relazione}
    Una relazione si definisce su due o più entità, e rappresenta un legame logico fra tali entità. Il numero di entità coinvolte in una relazione determina il suo grado (2, 3, 4, …)
    \begin{itemize}
        \item singolare
        \item sostantivi invece che verbi (se possibile e naturale)
    \end{itemize}
    \paragraph{Attributo di relazione}
    \begin{itemize}
        \item Un attributo di relazione è una proprietà locale di una relazione
        \item Un attributo della relazione $R$ tra le entita $E_1, E_2, \dots,E_n$descrive una proprietà che non è di $E1$, non è di $E2$,…, non è di $E_n$, ma del legame logico tra $E_1, E_2, \dots, E_n$ rappresentato da $R$
        \item è una funzione che associa ad ogni istanza di relazione un valore appartenente ad un insieme detto dominio dell’attributo
    \end{itemize}
    \newpage
    \paragraph{Vincolo di cardinalità}
    Un vincolo di cardinalità tra una entità E e una relazione R esprime un limite minimo (cardinalità minima) ed un limite massimo (cardinalità massima) di istanze della relazione R a cui può partecipare ogni istanza dell’entità E
    \paragraph{Relazione IS-A}
    Tra due classi rappresentate da due entità nello schema concettuale sussista la relazione IS-A (o relazione di sottoinsieme), e cioè che ogni istanza di una sia anche istanza dell’altra.
    \\La relazione IS-A nel modello ER si può definire tra due entità, che si dicono “entità padre” ed “entità figlia” (o sottoentità, cioè quella che rappresenta un sottoinsieme della entità padre).
    \paragraph{Generalizzazione}
    Talvolta, però, l’entità padre può generalizzare diverse sottoentità rispetto ad un unico criterio. In questo caso si parla di generalizzazione.
    \begin{itemize}
        \item \textbf{Completa}: l’unione delle istanze delle sottoentità è uguale all’insieme delle istanze dell’entità padre (freccia nera)
        \item \textbf{Non completa} (freccia vuota)
    \end{itemize}
    \paragraph{Attributi composti}
    Un attributo può anche essere definito su un dominio complesso. Di particolare interesse è il caso di dominio di tipo “record”, cioè composto di domini (anche detti coampi) elementari. Un attributo il cui dominio è di tipo record si dice composto.
    \paragraph{Relazioni n-arie}
    Una relazione di grado maggiore di 2 si dice n-aria
    \paragraph{Identificatori}
    Un identificatore di una entità è un insieme di proprietà (attributi o relazioni) che permettono di identificare univocamente le istanze di un’entità.
    \\Tipi:
    \begin{itemize}
        \item \textbf{Interno}, ossia formato solo da attributi di E
        \item \textbf{Esterno}, ossia formato da attributi di E e da relazioni che coinvolgono E, oppure solo da relazioni che coinvolgono E
    \end{itemize}
    \newpage
    \section{Modello Relazionale}
    \paragraph{Modello logico dei dati}
    Un insieme di strutture di rappresentazione utilizzabili per descrivere un insieme di dati, o schema logico, che, a sua volta, descrive una realtà di interesse
    \paragraph{Struttura di rappresentazione}
    Relazione $\rightarrow$ Descrizione nel modello logico \\\textit{es. Gara(Numero, Sede, SiglaAutomobile, Marca, Guidatore)}
    \paragraph{Schema logico}
    \begin{itemize}
        \item \textbf{Relationship}: che rappresenta una classe di fatti, nel modello Entity-Relationship
        \item \textbf{Relazione matematica}: come nella teoria degli insiemi
        \item \textbf{Relazione}: secondo il modello relazionale dei dati
    \end{itemize}

    \paragraph{Relazione matematica}
    \begin{itemize}
        \item $D_1, \dots, D_n$ ($n$  insiemi anche non distinti, detti anche domini) 
        \item Prodotto cartesiano $D_1 \times \dots \times D_n$: 
            \begin{itemize}
                \item L’insieme di tutte le  $n$-uple ($d_1, \dots, d_n$) tali che $d1 \in D_1, \dots, d_n \in D_n $
            \end{itemize}
        \item relazione matematica su $D_1, \dots, D_n$:
            \begin{itemize}
                \item un sottoinsieme di $D_1 \times \dots \times D_n$.
            \end{itemize}
        \item $D_1, \dots, D_n$ sono i domini della relazione 
    \end{itemize}
    Proprietà:
    \begin{itemize}
        \item una relazione matematica è un insieme di  $n$-uple al loro interno ordinate
        \item una relazione è un \textbf{insieme}, quindi:
        \begin{itemize}
            \item non c'è ordinamento fra le diverse n-uple
            \item le $n$-uple sono distinte (non ce ne possono essere due uguali) 
            \item ciascuna $n$-upla è al suo interno ordinata: l’$i$-esimo valore proviene dall’$i$-esimo dominio
        \end{itemize}
    \end{itemize}
    \paragraph{Relazioni}
    Una relazione (o tabella) nel modello relazionale rappresenta una relazione (in matematica) in cui:
    \begin{itemize}
        \item i valori di ogni colonna (attributo) sono fra loro omogenei (es. Juve, Lazio, ecc.)
        \item le righe (n-ple o tuple) sono diverse fra loro
        \item le intestazioni delle colonne (nomi di attributi) sono diverse tra loro 
    \end{itemize}
    In una relazione nel modello relazionale:
    \begin{itemize}
        \item l’ordinamento tra le righe è irrilevante
        \item l’ordinamento tra le colonne è irrilevante 
    \end{itemize}
    \paragraph{Modello relazionale basato su valori}
    Il modello relazionale è basato su valori. I riferimenti fra dati in relazioni diverse sono rappresentati per mezzo di valori dei domini che compaiono nelle ennuple. \\Vantaggi:
    \begin{itemize}
        \item l’utente finale vede gli stessi dati dei programmatori, che sono appunto costituiti dalle tabelle
        \item indipendenza della rappresentazione dalle strutture fisiche (ad es. indirizzi di memoria), che possono cambiare dinamicamente
        \item i dati sono portabili più facilmente da un sistema ad un altro, perché non dipendono dalle caratteristiche fisiche
    \end{itemize}
    \paragraph{Schema di relazione}
    Un nome $R$  con un insieme di attributi $A_1, \dots, A_n$: $$R(A_1, \dots, A_n)$$
    \paragraph{Schema di base di dati}
    Insieme di schemi di relazione: $$R = {R_1(X_1), \dots, R_k(X_k)}$$
    \paragraph{Ennupla / n-upla}
    \begin{itemize}
        \item Una ennupla (o n-pla) su un insieme di attributi $X$  è una funzione che associa a ciascun attributo $A$ in $X$  un valore del dominio di $A$
        \item Esempio $\rightarrow$ Codice:01 Titolo:Analisi Docente:Rossi
        \item Il simbolo $t[A]$ denota il valore della ennupla $t$ sull'attributo $A$
        \item Esempio $t[Codice]$ per la ennupla sopra è 01      
    \end{itemize}
    \paragraph{Istanza di relazione}
    Istanza di relazione (o relazione) su uno schema $R(X)$: insieme  $r$  di ennuple su $X$
    \paragraph{Istanza di base di dati}
    Istanza di base di dati (o base di dati) su uno schema  $R= \{R_1(X_1), \dots, R_n(X_n)\}$ : insieme di relazioni $r = {r_1, \dots, r_n}$ (con  $r_i$  relazione su  $R_i$) 
    \paragraph{Vincolo di integrità}
    Proprietà che deve essere soddisfatta da tutte le istanze di uno schema che rappresentano info corrette per l’applicazione\\
    Un vincolo di integrità è una funzione booleana (o predicato) che associa ad ogni istanza $r$:
    \begin{itemize}
        \item Il valore vero se la istanza è corretta (rappresentazione della realtà)
        \item Il valore falso se la istanza è scorretta (rappresentazione della realtà)
    \end{itemize}
    \paragraph{Vincolo intrarelazionale}
    Definiti all‘interno di una relazione
    \subparagraph{Vincolo di valore}
    Vincoli su valori (o di dominio), cioè su singoli valori di attributi  
    \subparagraph{Vincolo di ennupla}
    Vincoli di ennupla, cioè definiti sulle ennuple di una relazione
    \begin{itemize}
        \item Esprimono condizioni sui valori di ciascuna ennupla in una relazione
        \item Esempio: non ci possono essere due ennuple con lo stesso numero di matricola   
    \end{itemize}
    \subparagraph{Vincolo di relazione}
    Vincoli relativi all'insieme di ennuple di una relazione
    \subparagraph{Vincolo di chiave}
    Abbiamo necessità di individuare informazioni che  ci permettano di rappresentare ogni oggetto di interesse con una ennupla differente e identificarlo quando se ne abbia necessita
    \begin{itemize}
        \item \textbf{Chiave}: Insieme di attributi che identificano le ennuple di una relazione
        \item un insieme $K$ di attributi è \textbf{superchiave} per una relazione $r$  se $r$  non contiene due ennuple distinte $t_1$ e $t_2$  con  $t_1[K] = t_2[K]$
        \item $K$ è anche chiave per $r$ se è una superchiave minimale per $r$ (cioè non contiene un’altra superchiave)
    \end{itemize}
    \paragraph{Vincolo interrelazionale}
    Definiti tra due o più relazioni 
    \paragraph{Valori nulli}
    \begin{itemize}
        \item \textbf{valore sconosciuto}: non lo conosciamo ma esiste
        \item \textbf{valore inesistente}: non esiste 
        \item \textbf{valore senza informazione}: non si può dire niente (non lo conosciamo ovvero non esiste)
    \end{itemize}




    





\end{document}