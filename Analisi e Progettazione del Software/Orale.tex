\documentclass[12pt]{article}
\usepackage[italian]{babel}

\title{Orale APS}
\author{Giuseppe Facchi}

\begin{document}

\maketitle
\tableofcontents

\newpage

\section{Processi per lo sviluppo software}
\subsection{Introduzione}
Un processo per lo sviluppo software definisce un approccio per la costruzione, il rilascio e la manutenzione del software. Esempi:
\begin{itemize}
    \item \textbf{Processo a cascata}
    \item \textbf{UP}
    \item Scrum
    \item Spirale
\end{itemize}
\subsection{Processo a cascata}
Il processo software con ciclo di vita \textbf{a cascata} è basato su uno svolgimento \textbf{sequenziale} delle diverse attività.\\
Questo processo è \textbf{molto soggetto a fallimenti}, perché più si fa grande il software più è difficile implementare nuove features.

\subsection{Sviluppo iterativo ed evolutivo}
In questo approccio lo sviluppo è organizzato in una serie di mini-progetti brevi, di lunghezza fissa chiamati \textbf{iterazioni}. Il risultato di ciascuna iterazione è un \textbf{sistema eseguibile, testato e integrato}.\\
\textit{Può essere chiamato anche sviluppo iterativo e incrementale o sviluppo iterativo ed evolutivo}.
\paragraph{Vantaggi}
\begin{itemize}
    \item \textbf{Minore probabilità di fallimento}
    \item Riduzione \textbf{precoce} dei rischi maggiori
    \item \textbf{Progresso visibile} fin dall'inizio
    \item \textbf{Feedback precoce}, con coinvolgimento dell'utente
\end{itemize}
\paragraph{Timeboxing:} Le iterazioni hanno una \textbf{lunghezza fissata}
\subsubsection{Pianificazione iterativa, guidata dal rischio e dal cliente}
In ciascuna \textbf{iterazione} viene stabilito il piano di lavoro dettagliato per una sola iterazione. In \textit{UP} viene effettuata alla \textbf{fine} di ciascuna iterazione, per decidere il piano dell'iterazione successiva.
\subsection{UP}
\textbf{Unified Process} è un processo \textbf{iterativo} per la costruzione di sistemi orientati agli oggetti. In particolare viene ampiamente adottato \textbf{RUP (Rational Unified Process)}, un suo raffinamento.
\begin{itemize}
    \item Processo \textbf{iterativo}
    \item Le pratiche di UP forniscono un esempio di struttura rispetto a come \textbf{eseguire} e dunque come spiegare l'\textbf{OOA/D} \textit{(Object-Oriented Analysis/Design)}
    \item UP è \textbf{flessibile} e può essere applicato usando un \textbf{approccio leggero e agile} come ad esempio \textbf{Scrum}
\end{itemize}
\subsubsection{Le fasi di UP}
\begin{itemize}
    \item \textbf{Ideazione:} \textit{NON è la previsione dei requisiti del modello a cascata, ma una fase di fattibilità}
    \item \textbf{Elaborazione}
    \item \textbf{Costruzione}
    \item \textbf{Transizione}
\end{itemize}
\subsubsection{Le discipline di UP}
Una \textbf{disciplina} è un insieme di attività e dei relativi elaborati di una determinata area. In UP un elaborato è un qualsiasi prodotto di lavoro. Ci sono diverse discipline in UP:
\begin{itemize}
    \item \textbf{Modellazione del business}: Modello di dominio
    \item \textbf{Requisiti}: Modello dei Casi d'uso per definire requisiti funzionali e non funzionali
    \item \textbf{Progettazione}: Modello di Progetto
\end{itemize}
\subsection{Agile}
\textit{Non è possibile dare una definizione precisa di metodo agile perché le pratiche adottate variano notevolmente da metodo a metodo}. \\Una pratica di base è quella che prevede \textbf{iterazioni brevi}, con raffinamenti evolutivi dei piani, dei requisiti e del progetto. \\Lo \textbf{scopo della modellazione e dei modelli} è di \textbf{agevolare la comprensione e la comunicazione, NON di documentare}.
\subsection{Fase di Ideazione}
Lo scopo della fase di ideazione è stabilire una visione iniziale comune per gli obiettivi del progetto.
\begin{itemize}
    \item Non si definiscono tutti i requisiti nella fase di ideazione
    \item La maggior parte del'analisi dei requisiti avviene durante la fase di elaborazione
    \item Modello dei Casi d'Uso
\end{itemize}
\newpage
\section{Requisiti evolutivi}
Un sistema deve fornire un certo numero di funzionalità, relative alla gestione di alcune tipologie di informazione e al possesso di determinate qualità (sicurezza e prestazioni). Un requisito è una capacità o una condizione a cui il sistema deve essere conforme. 
\paragraph{Requisiti funzionali} Descrivono il comportamento del sistema in termini di funzionalità fornite ai suoi utenti. Possono essere espressi in forma di casi d'uso.
\paragraph{Requisiti non funzionali} Non riguardano le specifiche funzioni del sistema, ma sono relativi a proprietà del sistema, ad esempio sicurezza, prestazioni, scalabilità, ecc.
\subsection{Modello dei Casi d'uso}
In generale i casi d'uso sono storie scritte, testuali, di qualche attore che usa un sistema per raggiungere degli obiettivi. I casi d'uso non sono diagrammi, bensì testo.
\paragraph{Attore} Qualcosa o qualcuno dotato di comportamento, come una persona o un'organizzazione o un sistema informatico.
\paragraph{Scenario (istanza di caso d'uso)} E' una sequenza specifica di azioni e interazioni tra il sistema e alcuni attori.
\\[12pt]
\noindent Un caso d'uso è quindi una collezione di scenari correlati, sia di successo che di fallimento. I casi d'uso sono Requisiti Funzionali.
\subsubsection{Attori}
\begin{itemize}
    \item \textbf{Attore Primario}: Utilizza direttamente ii servizi del SuD (system under discussion) affinché vengano raggiunti gli obiettivi utente.
    \item \textbf{Attore Finale}: Vuole che il SuD sia utilizzato affinché vengano raggiunti dei suoi obiettivi. Spesso attore primario e finale coincidono (es. Cliente commercio elettronico).
    \item \textbf{Attore di Supporto}: Offre un servizio al SuD (es. Sistema di autorizzazione al pagamento).
    \item \textbf{Attore fuori scena}: Non è un attore primario, finale, di supporto (es. Governo). 
\end{itemize}
\subsubsection{Notazione: tre formati per i casi d'uso}
\begin{itemize}
    \item Formato breve
    \item Formato informale
    \item Formato dettagliato
\end{itemize}
\subsubsection{Come trovare i casi d'uso}
\begin{enumerate}
    \item Scegliere i confini del sistema
    \item Identificare gli attori primari
    \item Identificare gli obiettivi di ciascun attore primario
    \item Definire i casi d'uso che soddisfino gli obiettivi utente
\end{enumerate}
\subsubsection{Verificare l'utilità dei casi d'uso}
\paragraph{Test del capo} Il capo pone una domanda per cui ci sarà una risposta, se la risposta non soddisfa il capo il caso d'uso non è mirato a ottenere i risultati il cui valore sia misurabile. Non è però sempre vero (es. Autenticazione utente, concetto semplice ma difficilmente implementabile)
\paragraph{Test EBP (Elementary Business Process)} Simile al test del capo. Un processo di business elementare è un'attivitrà svolta da una persona in un determinato tempo e luogo, in risposta a un evento di business, che aggiunge valore e lascia i dati in uno stato consistente.
\paragraph{Test della dimensione} Un caso d'uso deve essere costituito da più passi.
\subparagraph{Esempi}
\begin{itemize}
    \item \textit{Negoziare un contratto con un fornitore}: Troppo ampio per essere un EBP
    \item \textit{Gestire una restituzione}: Passa il test del capo, è un EBP, le dimensioni vanno bene
    \item \textit{Effettuare il login}: Non passa il test del capo
    \item \textit{Spostare una pedina}: Non passa il test della dimensione
\end{itemize} 
\newpage
\section{Modellazione di dominio}
Rappresentazione visuale di classi concettuali o di oggetti del mondo reale e delle relazioni tra di essi.
\subsection{Strumenti per modellazione di dominio}
Applicando la notazione UML un modello di dominio può essere rappresentato da uno o più diagramma delle classi. Prevede:
\begin{itemize}
    \item \textbf{Classi concettuali}: rapprentano cose o concetti del dominio di interesse
    \item \textbf{Associazioni tra classi}: rappresentano relazioni tra oggetti di due classi
    \item \textbf{Attributi di classi concettuali}: rappresentano proprietà elementari degli oggetti di una classe
\end{itemize}
\subparagraph{Aggregazione e composizione}
\begin{itemize}
    \item \textbf{Aggregazione}: Tipo di associazione intero-parte (es. macchina-ruote)
    \item \textbf{Composizione}: Tipo di forte associazione intero-parte
    \begin{itemize}
        \item Ciascuna istanza della parte appartiene a una sola istanza dell'intero alla volta
        \item Ciascuna parte deve sempre appartenere ad un intero
        \item La vita delle parti è limitata dall'intero: le parti possono essere create dopo l'intero, ma non prima e possono essere distrutte prima dell'intero, ma non dopo
    \end{itemize}
\end{itemize}
\subsection{Diagramma degli oggetti}
Mostra un insieme di oggetti con i loro attributi e le loro relazioni in un dato momento.

\section{Diagrammi di interazione (parte di modello dei casi d'uso)}
\begin{itemize}
    \item \textbf{Illustrazione dei partecipanti con le Lifelines}: rappresentano un'istanza di una classe
    \item \textbf{Definizione dei messaggi scambiati tra gli oggetti}
\end{itemize}
\subsection{Operazioni di sistema e diagrammi di sequenza di sistema}
Un diagramma di sequenza di sistema è un elaborato che illustra, per un particolare caso d'uso, eventi di input e di output relativi ai sistemi in discussione con un formato "a steccato" in cui gli oggetti che partecipano all'interazione sono mostrati in alto, uno a fianco dell'altro. Esso costituisce un input per i contratti delle operazioni e soprattutto per la progettazione degli oggetti.
\\I casi d'uso descrivono il modo in cui gli attori esterni interagiscono con il sistema software che interessa creare. Durante questa interazione, un attore genera \textbf{eventi di sistema}, che costituiscono un input per il sistema, di solito per richiedere l'esecuzione di alcune \textbf{operazioni di sistema}, che sono operazioni che il sistema deve definire proprio per gestire tali eventi.
\subsubsection{Contratti}
Le sezioni di un contratto sono:
\begin{itemize}
    \item \textbf{Operazione}: Nome e parametri dell'operazione
    \item \textbf{Riferimenti}: Casi d'uso in cui può verificarsi questa operazione
    \item \textbf{Pre-condizioni}: Ipotesi sullo stato del sistema prima dell'esecuzione
    \item \textbf{Post-condizioni}: Descrive i cambiamenti di stato degli oggetti nel modello di dominio dopo il completamento dell'operazione
\end{itemize}
\subsection{Diagrammi di comunicazione}
Mostrano le interazioni tra gli oggetti in un formato a grafo o a rete. in cui gli oggetti possono essere posizionati dovunque nel diagramma.

\subsection{Pro/Contro diagrammi Sequenza/Comunicazione}
I diagrammi di sequenza sono strumenti più potenti perché:
\begin{itemize}
    \item UML è più incentrato sui diagrammi di sequenza
    \item Più facile vedere la sequenza "call-flow" poiché il tempo trascorre dall'alto verso il basso
\end{itemize}
I diagrammi di comunicazione sono strumenti più versatili perché:
\begin{itemize}
    \item Effettuare modifiche è più semplice senza modificare l'intero call-flow
    \item Sono più comodi da disegnare
\end{itemize}
\section{Diagramma delle classi di progetto}
\subsection{Generalizzazione}
Una generalizzazione è una \textbf{relazione tassonomica} tra un classificatore più generale e un classificatore più specifico. 
\\\textbf{NON equivale} all'ereditarietà nel \textbf{modello di dominio}
\\\textbf{Equivale} all'ereditarietà nel \textbf{modello di progettazione}, infatti qui la generalizzazione implica l'ereditarietà

\subsection{Aggregazione e composizione}
\begin{itemize}
    \item \textbf{Aggregazione}: Tipo di associazione intero-parte (es. macchina-ruote)
    \item \textbf{Composizione}: Tipo di forte associazione intero-parte
    \begin{itemize}
        \item Ciascuna istanza della parte appartiene a una sola istanza dell'intero alla volta
        \item Ciascuna parte deve sempre appartenere ad un intero
        \item La vita delle parti è limitata dall'intero: le parti possono essere create dopo l'intero, ma non prima e possono essere distrutte prima dell'intero, ma non dopo
    \end{itemize}
\end{itemize}
\section{GRASP: Progettazione di oggetti con responsabilità}


\end{document}